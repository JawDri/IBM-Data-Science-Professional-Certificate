\documentclass[11pt]{article}

    \usepackage[breakable]{tcolorbox}
    \usepackage{parskip} % Stop auto-indenting (to mimic markdown behaviour)
    
    \usepackage{iftex}
    \ifPDFTeX
    	\usepackage[T1]{fontenc}
    	\usepackage{mathpazo}
    \else
    	\usepackage{fontspec}
    \fi

    % Basic figure setup, for now with no caption control since it's done
    % automatically by Pandoc (which extracts ![](path) syntax from Markdown).
    \usepackage{graphicx}
    % Maintain compatibility with old templates. Remove in nbconvert 6.0
    \let\Oldincludegraphics\includegraphics
    % Ensure that by default, figures have no caption (until we provide a
    % proper Figure object with a Caption API and a way to capture that
    % in the conversion process - todo).
    \usepackage{caption}
    \DeclareCaptionFormat{nocaption}{}
    \captionsetup{format=nocaption,aboveskip=0pt,belowskip=0pt}

    \usepackage[Export]{adjustbox} % Used to constrain images to a maximum size
    \adjustboxset{max size={0.9\linewidth}{0.9\paperheight}}
    \usepackage{float}
    \floatplacement{figure}{H} % forces figures to be placed at the correct location
    \usepackage{xcolor} % Allow colors to be defined
    \usepackage{enumerate} % Needed for markdown enumerations to work
    \usepackage{geometry} % Used to adjust the document margins
    \usepackage{amsmath} % Equations
    \usepackage{amssymb} % Equations
    \usepackage{textcomp} % defines textquotesingle
    % Hack from http://tex.stackexchange.com/a/47451/13684:
    \AtBeginDocument{%
        \def\PYZsq{\textquotesingle}% Upright quotes in Pygmentized code
    }
    \usepackage{upquote} % Upright quotes for verbatim code
    \usepackage{eurosym} % defines \euro
    \usepackage[mathletters]{ucs} % Extended unicode (utf-8) support
    \usepackage{fancyvrb} % verbatim replacement that allows latex
    \usepackage{grffile} % extends the file name processing of package graphics 
                         % to support a larger range
    \makeatletter % fix for grffile with XeLaTeX
    \def\Gread@@xetex#1{%
      \IfFileExists{"\Gin@base".bb}%
      {\Gread@eps{\Gin@base.bb}}%
      {\Gread@@xetex@aux#1}%
    }
    \makeatother

    % The hyperref package gives us a pdf with properly built
    % internal navigation ('pdf bookmarks' for the table of contents,
    % internal cross-reference links, web links for URLs, etc.)
    \usepackage{hyperref}
    % The default LaTeX title has an obnoxious amount of whitespace. By default,
    % titling removes some of it. It also provides customization options.
    \usepackage{titling}
    \usepackage{longtable} % longtable support required by pandoc >1.10
    \usepackage{booktabs}  % table support for pandoc > 1.12.2
    \usepackage[inline]{enumitem} % IRkernel/repr support (it uses the enumerate* environment)
    \usepackage[normalem]{ulem} % ulem is needed to support strikethroughs (\sout)
                                % normalem makes italics be italics, not underlines
    \usepackage{mathrsfs}
    

    
    % Colors for the hyperref package
    \definecolor{urlcolor}{rgb}{0,.145,.698}
    \definecolor{linkcolor}{rgb}{.71,0.21,0.01}
    \definecolor{citecolor}{rgb}{.12,.54,.11}

    % ANSI colors
    \definecolor{ansi-black}{HTML}{3E424D}
    \definecolor{ansi-black-intense}{HTML}{282C36}
    \definecolor{ansi-red}{HTML}{E75C58}
    \definecolor{ansi-red-intense}{HTML}{B22B31}
    \definecolor{ansi-green}{HTML}{00A250}
    \definecolor{ansi-green-intense}{HTML}{007427}
    \definecolor{ansi-yellow}{HTML}{DDB62B}
    \definecolor{ansi-yellow-intense}{HTML}{B27D12}
    \definecolor{ansi-blue}{HTML}{208FFB}
    \definecolor{ansi-blue-intense}{HTML}{0065CA}
    \definecolor{ansi-magenta}{HTML}{D160C4}
    \definecolor{ansi-magenta-intense}{HTML}{A03196}
    \definecolor{ansi-cyan}{HTML}{60C6C8}
    \definecolor{ansi-cyan-intense}{HTML}{258F8F}
    \definecolor{ansi-white}{HTML}{C5C1B4}
    \definecolor{ansi-white-intense}{HTML}{A1A6B2}
    \definecolor{ansi-default-inverse-fg}{HTML}{FFFFFF}
    \definecolor{ansi-default-inverse-bg}{HTML}{000000}

    % commands and environments needed by pandoc snippets
    % extracted from the output of `pandoc -s`
    \providecommand{\tightlist}{%
      \setlength{\itemsep}{0pt}\setlength{\parskip}{0pt}}
    \DefineVerbatimEnvironment{Highlighting}{Verbatim}{commandchars=\\\{\}}
    % Add ',fontsize=\small' for more characters per line
    \newenvironment{Shaded}{}{}
    \newcommand{\KeywordTok}[1]{\textcolor[rgb]{0.00,0.44,0.13}{\textbf{{#1}}}}
    \newcommand{\DataTypeTok}[1]{\textcolor[rgb]{0.56,0.13,0.00}{{#1}}}
    \newcommand{\DecValTok}[1]{\textcolor[rgb]{0.25,0.63,0.44}{{#1}}}
    \newcommand{\BaseNTok}[1]{\textcolor[rgb]{0.25,0.63,0.44}{{#1}}}
    \newcommand{\FloatTok}[1]{\textcolor[rgb]{0.25,0.63,0.44}{{#1}}}
    \newcommand{\CharTok}[1]{\textcolor[rgb]{0.25,0.44,0.63}{{#1}}}
    \newcommand{\StringTok}[1]{\textcolor[rgb]{0.25,0.44,0.63}{{#1}}}
    \newcommand{\CommentTok}[1]{\textcolor[rgb]{0.38,0.63,0.69}{\textit{{#1}}}}
    \newcommand{\OtherTok}[1]{\textcolor[rgb]{0.00,0.44,0.13}{{#1}}}
    \newcommand{\AlertTok}[1]{\textcolor[rgb]{1.00,0.00,0.00}{\textbf{{#1}}}}
    \newcommand{\FunctionTok}[1]{\textcolor[rgb]{0.02,0.16,0.49}{{#1}}}
    \newcommand{\RegionMarkerTok}[1]{{#1}}
    \newcommand{\ErrorTok}[1]{\textcolor[rgb]{1.00,0.00,0.00}{\textbf{{#1}}}}
    \newcommand{\NormalTok}[1]{{#1}}
    
    % Additional commands for more recent versions of Pandoc
    \newcommand{\ConstantTok}[1]{\textcolor[rgb]{0.53,0.00,0.00}{{#1}}}
    \newcommand{\SpecialCharTok}[1]{\textcolor[rgb]{0.25,0.44,0.63}{{#1}}}
    \newcommand{\VerbatimStringTok}[1]{\textcolor[rgb]{0.25,0.44,0.63}{{#1}}}
    \newcommand{\SpecialStringTok}[1]{\textcolor[rgb]{0.73,0.40,0.53}{{#1}}}
    \newcommand{\ImportTok}[1]{{#1}}
    \newcommand{\DocumentationTok}[1]{\textcolor[rgb]{0.73,0.13,0.13}{\textit{{#1}}}}
    \newcommand{\AnnotationTok}[1]{\textcolor[rgb]{0.38,0.63,0.69}{\textbf{\textit{{#1}}}}}
    \newcommand{\CommentVarTok}[1]{\textcolor[rgb]{0.38,0.63,0.69}{\textbf{\textit{{#1}}}}}
    \newcommand{\VariableTok}[1]{\textcolor[rgb]{0.10,0.09,0.49}{{#1}}}
    \newcommand{\ControlFlowTok}[1]{\textcolor[rgb]{0.00,0.44,0.13}{\textbf{{#1}}}}
    \newcommand{\OperatorTok}[1]{\textcolor[rgb]{0.40,0.40,0.40}{{#1}}}
    \newcommand{\BuiltInTok}[1]{{#1}}
    \newcommand{\ExtensionTok}[1]{{#1}}
    \newcommand{\PreprocessorTok}[1]{\textcolor[rgb]{0.74,0.48,0.00}{{#1}}}
    \newcommand{\AttributeTok}[1]{\textcolor[rgb]{0.49,0.56,0.16}{{#1}}}
    \newcommand{\InformationTok}[1]{\textcolor[rgb]{0.38,0.63,0.69}{\textbf{\textit{{#1}}}}}
    \newcommand{\WarningTok}[1]{\textcolor[rgb]{0.38,0.63,0.69}{\textbf{\textit{{#1}}}}}
    
    
    % Define a nice break command that doesn't care if a line doesn't already
    % exist.
    \def\br{\hspace*{\fill} \\* }
    % Math Jax compatibility definitions
    \def\gt{>}
    \def\lt{<}
    \let\Oldtex\TeX
    \let\Oldlatex\LaTeX
    \renewcommand{\TeX}{\textrm{\Oldtex}}
    \renewcommand{\LaTeX}{\textrm{\Oldlatex}}
    % Document parameters
    % Document title
    \title{review-introduction}
    
    
    
    
    
% Pygments definitions
\makeatletter
\def\PY@reset{\let\PY@it=\relax \let\PY@bf=\relax%
    \let\PY@ul=\relax \let\PY@tc=\relax%
    \let\PY@bc=\relax \let\PY@ff=\relax}
\def\PY@tok#1{\csname PY@tok@#1\endcsname}
\def\PY@toks#1+{\ifx\relax#1\empty\else%
    \PY@tok{#1}\expandafter\PY@toks\fi}
\def\PY@do#1{\PY@bc{\PY@tc{\PY@ul{%
    \PY@it{\PY@bf{\PY@ff{#1}}}}}}}
\def\PY#1#2{\PY@reset\PY@toks#1+\relax+\PY@do{#2}}

\expandafter\def\csname PY@tok@w\endcsname{\def\PY@tc##1{\textcolor[rgb]{0.73,0.73,0.73}{##1}}}
\expandafter\def\csname PY@tok@c\endcsname{\let\PY@it=\textit\def\PY@tc##1{\textcolor[rgb]{0.25,0.50,0.50}{##1}}}
\expandafter\def\csname PY@tok@cp\endcsname{\def\PY@tc##1{\textcolor[rgb]{0.74,0.48,0.00}{##1}}}
\expandafter\def\csname PY@tok@k\endcsname{\let\PY@bf=\textbf\def\PY@tc##1{\textcolor[rgb]{0.00,0.50,0.00}{##1}}}
\expandafter\def\csname PY@tok@kp\endcsname{\def\PY@tc##1{\textcolor[rgb]{0.00,0.50,0.00}{##1}}}
\expandafter\def\csname PY@tok@kt\endcsname{\def\PY@tc##1{\textcolor[rgb]{0.69,0.00,0.25}{##1}}}
\expandafter\def\csname PY@tok@o\endcsname{\def\PY@tc##1{\textcolor[rgb]{0.40,0.40,0.40}{##1}}}
\expandafter\def\csname PY@tok@ow\endcsname{\let\PY@bf=\textbf\def\PY@tc##1{\textcolor[rgb]{0.67,0.13,1.00}{##1}}}
\expandafter\def\csname PY@tok@nb\endcsname{\def\PY@tc##1{\textcolor[rgb]{0.00,0.50,0.00}{##1}}}
\expandafter\def\csname PY@tok@nf\endcsname{\def\PY@tc##1{\textcolor[rgb]{0.00,0.00,1.00}{##1}}}
\expandafter\def\csname PY@tok@nc\endcsname{\let\PY@bf=\textbf\def\PY@tc##1{\textcolor[rgb]{0.00,0.00,1.00}{##1}}}
\expandafter\def\csname PY@tok@nn\endcsname{\let\PY@bf=\textbf\def\PY@tc##1{\textcolor[rgb]{0.00,0.00,1.00}{##1}}}
\expandafter\def\csname PY@tok@ne\endcsname{\let\PY@bf=\textbf\def\PY@tc##1{\textcolor[rgb]{0.82,0.25,0.23}{##1}}}
\expandafter\def\csname PY@tok@nv\endcsname{\def\PY@tc##1{\textcolor[rgb]{0.10,0.09,0.49}{##1}}}
\expandafter\def\csname PY@tok@no\endcsname{\def\PY@tc##1{\textcolor[rgb]{0.53,0.00,0.00}{##1}}}
\expandafter\def\csname PY@tok@nl\endcsname{\def\PY@tc##1{\textcolor[rgb]{0.63,0.63,0.00}{##1}}}
\expandafter\def\csname PY@tok@ni\endcsname{\let\PY@bf=\textbf\def\PY@tc##1{\textcolor[rgb]{0.60,0.60,0.60}{##1}}}
\expandafter\def\csname PY@tok@na\endcsname{\def\PY@tc##1{\textcolor[rgb]{0.49,0.56,0.16}{##1}}}
\expandafter\def\csname PY@tok@nt\endcsname{\let\PY@bf=\textbf\def\PY@tc##1{\textcolor[rgb]{0.00,0.50,0.00}{##1}}}
\expandafter\def\csname PY@tok@nd\endcsname{\def\PY@tc##1{\textcolor[rgb]{0.67,0.13,1.00}{##1}}}
\expandafter\def\csname PY@tok@s\endcsname{\def\PY@tc##1{\textcolor[rgb]{0.73,0.13,0.13}{##1}}}
\expandafter\def\csname PY@tok@sd\endcsname{\let\PY@it=\textit\def\PY@tc##1{\textcolor[rgb]{0.73,0.13,0.13}{##1}}}
\expandafter\def\csname PY@tok@si\endcsname{\let\PY@bf=\textbf\def\PY@tc##1{\textcolor[rgb]{0.73,0.40,0.53}{##1}}}
\expandafter\def\csname PY@tok@se\endcsname{\let\PY@bf=\textbf\def\PY@tc##1{\textcolor[rgb]{0.73,0.40,0.13}{##1}}}
\expandafter\def\csname PY@tok@sr\endcsname{\def\PY@tc##1{\textcolor[rgb]{0.73,0.40,0.53}{##1}}}
\expandafter\def\csname PY@tok@ss\endcsname{\def\PY@tc##1{\textcolor[rgb]{0.10,0.09,0.49}{##1}}}
\expandafter\def\csname PY@tok@sx\endcsname{\def\PY@tc##1{\textcolor[rgb]{0.00,0.50,0.00}{##1}}}
\expandafter\def\csname PY@tok@m\endcsname{\def\PY@tc##1{\textcolor[rgb]{0.40,0.40,0.40}{##1}}}
\expandafter\def\csname PY@tok@gh\endcsname{\let\PY@bf=\textbf\def\PY@tc##1{\textcolor[rgb]{0.00,0.00,0.50}{##1}}}
\expandafter\def\csname PY@tok@gu\endcsname{\let\PY@bf=\textbf\def\PY@tc##1{\textcolor[rgb]{0.50,0.00,0.50}{##1}}}
\expandafter\def\csname PY@tok@gd\endcsname{\def\PY@tc##1{\textcolor[rgb]{0.63,0.00,0.00}{##1}}}
\expandafter\def\csname PY@tok@gi\endcsname{\def\PY@tc##1{\textcolor[rgb]{0.00,0.63,0.00}{##1}}}
\expandafter\def\csname PY@tok@gr\endcsname{\def\PY@tc##1{\textcolor[rgb]{1.00,0.00,0.00}{##1}}}
\expandafter\def\csname PY@tok@ge\endcsname{\let\PY@it=\textit}
\expandafter\def\csname PY@tok@gs\endcsname{\let\PY@bf=\textbf}
\expandafter\def\csname PY@tok@gp\endcsname{\let\PY@bf=\textbf\def\PY@tc##1{\textcolor[rgb]{0.00,0.00,0.50}{##1}}}
\expandafter\def\csname PY@tok@go\endcsname{\def\PY@tc##1{\textcolor[rgb]{0.53,0.53,0.53}{##1}}}
\expandafter\def\csname PY@tok@gt\endcsname{\def\PY@tc##1{\textcolor[rgb]{0.00,0.27,0.87}{##1}}}
\expandafter\def\csname PY@tok@err\endcsname{\def\PY@bc##1{\setlength{\fboxsep}{0pt}\fcolorbox[rgb]{1.00,0.00,0.00}{1,1,1}{\strut ##1}}}
\expandafter\def\csname PY@tok@kc\endcsname{\let\PY@bf=\textbf\def\PY@tc##1{\textcolor[rgb]{0.00,0.50,0.00}{##1}}}
\expandafter\def\csname PY@tok@kd\endcsname{\let\PY@bf=\textbf\def\PY@tc##1{\textcolor[rgb]{0.00,0.50,0.00}{##1}}}
\expandafter\def\csname PY@tok@kn\endcsname{\let\PY@bf=\textbf\def\PY@tc##1{\textcolor[rgb]{0.00,0.50,0.00}{##1}}}
\expandafter\def\csname PY@tok@kr\endcsname{\let\PY@bf=\textbf\def\PY@tc##1{\textcolor[rgb]{0.00,0.50,0.00}{##1}}}
\expandafter\def\csname PY@tok@bp\endcsname{\def\PY@tc##1{\textcolor[rgb]{0.00,0.50,0.00}{##1}}}
\expandafter\def\csname PY@tok@fm\endcsname{\def\PY@tc##1{\textcolor[rgb]{0.00,0.00,1.00}{##1}}}
\expandafter\def\csname PY@tok@vc\endcsname{\def\PY@tc##1{\textcolor[rgb]{0.10,0.09,0.49}{##1}}}
\expandafter\def\csname PY@tok@vg\endcsname{\def\PY@tc##1{\textcolor[rgb]{0.10,0.09,0.49}{##1}}}
\expandafter\def\csname PY@tok@vi\endcsname{\def\PY@tc##1{\textcolor[rgb]{0.10,0.09,0.49}{##1}}}
\expandafter\def\csname PY@tok@vm\endcsname{\def\PY@tc##1{\textcolor[rgb]{0.10,0.09,0.49}{##1}}}
\expandafter\def\csname PY@tok@sa\endcsname{\def\PY@tc##1{\textcolor[rgb]{0.73,0.13,0.13}{##1}}}
\expandafter\def\csname PY@tok@sb\endcsname{\def\PY@tc##1{\textcolor[rgb]{0.73,0.13,0.13}{##1}}}
\expandafter\def\csname PY@tok@sc\endcsname{\def\PY@tc##1{\textcolor[rgb]{0.73,0.13,0.13}{##1}}}
\expandafter\def\csname PY@tok@dl\endcsname{\def\PY@tc##1{\textcolor[rgb]{0.73,0.13,0.13}{##1}}}
\expandafter\def\csname PY@tok@s2\endcsname{\def\PY@tc##1{\textcolor[rgb]{0.73,0.13,0.13}{##1}}}
\expandafter\def\csname PY@tok@sh\endcsname{\def\PY@tc##1{\textcolor[rgb]{0.73,0.13,0.13}{##1}}}
\expandafter\def\csname PY@tok@s1\endcsname{\def\PY@tc##1{\textcolor[rgb]{0.73,0.13,0.13}{##1}}}
\expandafter\def\csname PY@tok@mb\endcsname{\def\PY@tc##1{\textcolor[rgb]{0.40,0.40,0.40}{##1}}}
\expandafter\def\csname PY@tok@mf\endcsname{\def\PY@tc##1{\textcolor[rgb]{0.40,0.40,0.40}{##1}}}
\expandafter\def\csname PY@tok@mh\endcsname{\def\PY@tc##1{\textcolor[rgb]{0.40,0.40,0.40}{##1}}}
\expandafter\def\csname PY@tok@mi\endcsname{\def\PY@tc##1{\textcolor[rgb]{0.40,0.40,0.40}{##1}}}
\expandafter\def\csname PY@tok@il\endcsname{\def\PY@tc##1{\textcolor[rgb]{0.40,0.40,0.40}{##1}}}
\expandafter\def\csname PY@tok@mo\endcsname{\def\PY@tc##1{\textcolor[rgb]{0.40,0.40,0.40}{##1}}}
\expandafter\def\csname PY@tok@ch\endcsname{\let\PY@it=\textit\def\PY@tc##1{\textcolor[rgb]{0.25,0.50,0.50}{##1}}}
\expandafter\def\csname PY@tok@cm\endcsname{\let\PY@it=\textit\def\PY@tc##1{\textcolor[rgb]{0.25,0.50,0.50}{##1}}}
\expandafter\def\csname PY@tok@cpf\endcsname{\let\PY@it=\textit\def\PY@tc##1{\textcolor[rgb]{0.25,0.50,0.50}{##1}}}
\expandafter\def\csname PY@tok@c1\endcsname{\let\PY@it=\textit\def\PY@tc##1{\textcolor[rgb]{0.25,0.50,0.50}{##1}}}
\expandafter\def\csname PY@tok@cs\endcsname{\let\PY@it=\textit\def\PY@tc##1{\textcolor[rgb]{0.25,0.50,0.50}{##1}}}

\def\PYZbs{\char`\\}
\def\PYZus{\char`\_}
\def\PYZob{\char`\{}
\def\PYZcb{\char`\}}
\def\PYZca{\char`\^}
\def\PYZam{\char`\&}
\def\PYZlt{\char`\<}
\def\PYZgt{\char`\>}
\def\PYZsh{\char`\#}
\def\PYZpc{\char`\%}
\def\PYZdl{\char`\$}
\def\PYZhy{\char`\-}
\def\PYZsq{\char`\'}
\def\PYZdq{\char`\"}
\def\PYZti{\char`\~}
% for compatibility with earlier versions
\def\PYZat{@}
\def\PYZlb{[}
\def\PYZrb{]}
\makeatother


    % For linebreaks inside Verbatim environment from package fancyvrb. 
    \makeatletter
        \newbox\Wrappedcontinuationbox 
        \newbox\Wrappedvisiblespacebox 
        \newcommand*\Wrappedvisiblespace {\textcolor{red}{\textvisiblespace}} 
        \newcommand*\Wrappedcontinuationsymbol {\textcolor{red}{\llap{\tiny$\m@th\hookrightarrow$}}} 
        \newcommand*\Wrappedcontinuationindent {3ex } 
        \newcommand*\Wrappedafterbreak {\kern\Wrappedcontinuationindent\copy\Wrappedcontinuationbox} 
        % Take advantage of the already applied Pygments mark-up to insert 
        % potential linebreaks for TeX processing. 
        %        {, <, #, %, $, ' and ": go to next line. 
        %        _, }, ^, &, >, - and ~: stay at end of broken line. 
        % Use of \textquotesingle for straight quote. 
        \newcommand*\Wrappedbreaksatspecials {% 
            \def\PYGZus{\discretionary{\char`\_}{\Wrappedafterbreak}{\char`\_}}% 
            \def\PYGZob{\discretionary{}{\Wrappedafterbreak\char`\{}{\char`\{}}% 
            \def\PYGZcb{\discretionary{\char`\}}{\Wrappedafterbreak}{\char`\}}}% 
            \def\PYGZca{\discretionary{\char`\^}{\Wrappedafterbreak}{\char`\^}}% 
            \def\PYGZam{\discretionary{\char`\&}{\Wrappedafterbreak}{\char`\&}}% 
            \def\PYGZlt{\discretionary{}{\Wrappedafterbreak\char`\<}{\char`\<}}% 
            \def\PYGZgt{\discretionary{\char`\>}{\Wrappedafterbreak}{\char`\>}}% 
            \def\PYGZsh{\discretionary{}{\Wrappedafterbreak\char`\#}{\char`\#}}% 
            \def\PYGZpc{\discretionary{}{\Wrappedafterbreak\char`\%}{\char`\%}}% 
            \def\PYGZdl{\discretionary{}{\Wrappedafterbreak\char`\$}{\char`\$}}% 
            \def\PYGZhy{\discretionary{\char`\-}{\Wrappedafterbreak}{\char`\-}}% 
            \def\PYGZsq{\discretionary{}{\Wrappedafterbreak\textquotesingle}{\textquotesingle}}% 
            \def\PYGZdq{\discretionary{}{\Wrappedafterbreak\char`\"}{\char`\"}}% 
            \def\PYGZti{\discretionary{\char`\~}{\Wrappedafterbreak}{\char`\~}}% 
        } 
        % Some characters . , ; ? ! / are not pygmentized. 
        % This macro makes them "active" and they will insert potential linebreaks 
        \newcommand*\Wrappedbreaksatpunct {% 
            \lccode`\~`\.\lowercase{\def~}{\discretionary{\hbox{\char`\.}}{\Wrappedafterbreak}{\hbox{\char`\.}}}% 
            \lccode`\~`\,\lowercase{\def~}{\discretionary{\hbox{\char`\,}}{\Wrappedafterbreak}{\hbox{\char`\,}}}% 
            \lccode`\~`\;\lowercase{\def~}{\discretionary{\hbox{\char`\;}}{\Wrappedafterbreak}{\hbox{\char`\;}}}% 
            \lccode`\~`\:\lowercase{\def~}{\discretionary{\hbox{\char`\:}}{\Wrappedafterbreak}{\hbox{\char`\:}}}% 
            \lccode`\~`\?\lowercase{\def~}{\discretionary{\hbox{\char`\?}}{\Wrappedafterbreak}{\hbox{\char`\?}}}% 
            \lccode`\~`\!\lowercase{\def~}{\discretionary{\hbox{\char`\!}}{\Wrappedafterbreak}{\hbox{\char`\!}}}% 
            \lccode`\~`\/\lowercase{\def~}{\discretionary{\hbox{\char`\/}}{\Wrappedafterbreak}{\hbox{\char`\/}}}% 
            \catcode`\.\active
            \catcode`\,\active 
            \catcode`\;\active
            \catcode`\:\active
            \catcode`\?\active
            \catcode`\!\active
            \catcode`\/\active 
            \lccode`\~`\~ 	
        }
    \makeatother

    \let\OriginalVerbatim=\Verbatim
    \makeatletter
    \renewcommand{\Verbatim}[1][1]{%
        %\parskip\z@skip
        \sbox\Wrappedcontinuationbox {\Wrappedcontinuationsymbol}%
        \sbox\Wrappedvisiblespacebox {\FV@SetupFont\Wrappedvisiblespace}%
        \def\FancyVerbFormatLine ##1{\hsize\linewidth
            \vtop{\raggedright\hyphenpenalty\z@\exhyphenpenalty\z@
                \doublehyphendemerits\z@\finalhyphendemerits\z@
                \strut ##1\strut}%
        }%
        % If the linebreak is at a space, the latter will be displayed as visible
        % space at end of first line, and a continuation symbol starts next line.
        % Stretch/shrink are however usually zero for typewriter font.
        \def\FV@Space {%
            \nobreak\hskip\z@ plus\fontdimen3\font minus\fontdimen4\font
            \discretionary{\copy\Wrappedvisiblespacebox}{\Wrappedafterbreak}
            {\kern\fontdimen2\font}%
        }%
        
        % Allow breaks at special characters using \PYG... macros.
        \Wrappedbreaksatspecials
        % Breaks at punctuation characters . , ; ? ! and / need catcode=\active 	
        \OriginalVerbatim[#1,codes*=\Wrappedbreaksatpunct]%
    }
    \makeatother

    % Exact colors from NB
    \definecolor{incolor}{HTML}{303F9F}
    \definecolor{outcolor}{HTML}{D84315}
    \definecolor{cellborder}{HTML}{CFCFCF}
    \definecolor{cellbackground}{HTML}{F7F7F7}
    
    % prompt
    \makeatletter
    \newcommand{\boxspacing}{\kern\kvtcb@left@rule\kern\kvtcb@boxsep}
    \makeatother
    \newcommand{\prompt}[4]{
        \ttfamily\llap{{\color{#2}[#3]:\hspace{3pt}#4}}\vspace{-\baselineskip}
    }
    

    
    % Prevent overflowing lines due to hard-to-break entities
    \sloppy 
    % Setup hyperref package
    \hypersetup{
      breaklinks=true,  % so long urls are correctly broken across lines
      colorlinks=true,
      urlcolor=urlcolor,
      linkcolor=linkcolor,
      citecolor=citecolor,
      }
    % Slightly bigger margins than the latex defaults
    
    \geometry{verbose,tmargin=1in,bmargin=1in,lmargin=1in,rmargin=1in}
    
    

\begin{document}
    
    \maketitle
    
    

    
    \begin{verbatim}
<a href="https://cocl.us/corsera_da0101en_notebook_top">
     <img src="https://s3-api.us-geo.objectstorage.softlayer.net/cf-courses-data/CognitiveClass/DA0101EN/Images/TopAd.png" width="750" align="center">
</a>
\end{verbatim}

    Data Analysis with Python

    Introduction

Welcome!

In this section, you will learn how to approach data acquisition in
various ways, and obtain necessary insights from a dataset. By the end
of this lab, you will successfully load the data into Jupyter Notebook,
and gain some fundamental insights via Pandas Library.

    Table of Contents

Data Acquisition

Basic Insight of Dataset

Estimated Time Needed: 10 min

    Data Acquisition

There are various formats for a dataset, .csv, .json, .xlsx etc. The
dataset can be stored in different places, on your local machine or
sometimes online. In this section, you will learn how to load a dataset
into our Jupyter Notebook. In our case, the Automobile Dataset is an
online source, and it is in CSV (comma separated value) format. Let's
use this dataset as an example to practice data reading.

data source:
https://archive.ics.uci.edu/ml/machine-learning-databases/autos/imports-85.data

data type: csv

The Pandas Library is a useful tool that enables us to read various
datasets into a data frame; our Jupyter notebook platforms have a
built-in Pandas Library so that all we need to do is import Pandas
without installing.

    \begin{tcolorbox}[breakable, size=fbox, boxrule=1pt, pad at break*=1mm,colback=cellbackground, colframe=cellborder]
\prompt{In}{incolor}{1}{\boxspacing}
\begin{Verbatim}[commandchars=\\\{\}]
\PY{c+c1}{\PYZsh{} import pandas library}
\PY{k+kn}{import} \PY{n+nn}{pandas} \PY{k}{as} \PY{n+nn}{pd}
\end{Verbatim}
\end{tcolorbox}

    Read Data

We use pandas.read\_csv() function to read the csv file. In the bracket,
we put the file path along with a quotation mark, so that pandas will
read the file into a data frame from that address. The file path can be
either an URL or your local file address. Because the data does not
include headers, we can add an argument headers = None inside the
read\_csv() method, so that pandas will not automatically set the first
row as a header. You can also assign the dataset to any variable you
create.

    This dataset was hosted on IBM Cloud object click HERE for free storage.

    \begin{tcolorbox}[breakable, size=fbox, boxrule=1pt, pad at break*=1mm,colback=cellbackground, colframe=cellborder]
\prompt{In}{incolor}{2}{\boxspacing}
\begin{Verbatim}[commandchars=\\\{\}]
\PY{c+c1}{\PYZsh{} Import pandas library}
\PY{k+kn}{import} \PY{n+nn}{pandas} \PY{k}{as} \PY{n+nn}{pd}

\PY{c+c1}{\PYZsh{} Read the online file by the URL provides above, and assign it to variable \PYZdq{}df\PYZdq{}}
\PY{n}{other\PYZus{}path} \PY{o}{=} \PY{l+s+s2}{\PYZdq{}}\PY{l+s+s2}{https://s3\PYZhy{}api.us\PYZhy{}geo.objectstorage.softlayer.net/cf\PYZhy{}courses\PYZhy{}data/CognitiveClass/DA0101EN/auto.csv}\PY{l+s+s2}{\PYZdq{}}
\PY{n}{df} \PY{o}{=} \PY{n}{pd}\PY{o}{.}\PY{n}{read\PYZus{}csv}\PY{p}{(}\PY{n}{other\PYZus{}path}\PY{p}{,} \PY{n}{header}\PY{o}{=}\PY{k+kc}{None}\PY{p}{)}
\end{Verbatim}
\end{tcolorbox}

    After reading the dataset, we can use the dataframe.head(n) method to
check the top n rows of the dataframe; where n is an integer. Contrary
to dataframe.head(n), dataframe.tail(n) will show you the bottom n rows
of the dataframe.

    \begin{tcolorbox}[breakable, size=fbox, boxrule=1pt, pad at break*=1mm,colback=cellbackground, colframe=cellborder]
\prompt{In}{incolor}{3}{\boxspacing}
\begin{Verbatim}[commandchars=\\\{\}]
\PY{c+c1}{\PYZsh{} show the first 5 rows using dataframe.head() method}
\PY{n+nb}{print}\PY{p}{(}\PY{l+s+s2}{\PYZdq{}}\PY{l+s+s2}{The first 5 rows of the dataframe}\PY{l+s+s2}{\PYZdq{}}\PY{p}{)} 
\PY{n}{df}\PY{o}{.}\PY{n}{head}\PY{p}{(}\PY{l+m+mi}{5}\PY{p}{)}
\end{Verbatim}
\end{tcolorbox}

    \begin{Verbatim}[commandchars=\\\{\}]
The first 5 rows of the dataframe
    \end{Verbatim}

            \begin{tcolorbox}[breakable, size=fbox, boxrule=.5pt, pad at break*=1mm, opacityfill=0]
\prompt{Out}{outcolor}{3}{\boxspacing}
\begin{Verbatim}[commandchars=\\\{\}]
   0    1            2    3    4     5            6    7      8     9   {\ldots}  \textbackslash{}
0   3    ?  alfa-romero  gas  std   two  convertible  rwd  front  88.6  {\ldots}
1   3    ?  alfa-romero  gas  std   two  convertible  rwd  front  88.6  {\ldots}
2   1    ?  alfa-romero  gas  std   two    hatchback  rwd  front  94.5  {\ldots}
3   2  164         audi  gas  std  four        sedan  fwd  front  99.8  {\ldots}
4   2  164         audi  gas  std  four        sedan  4wd  front  99.4  {\ldots}

    16    17    18    19    20   21    22  23  24     25
0  130  mpfi  3.47  2.68   9.0  111  5000  21  27  13495
1  130  mpfi  3.47  2.68   9.0  111  5000  21  27  16500
2  152  mpfi  2.68  3.47   9.0  154  5000  19  26  16500
3  109  mpfi  3.19  3.40  10.0  102  5500  24  30  13950
4  136  mpfi  3.19  3.40   8.0  115  5500  18  22  17450

[5 rows x 26 columns]
\end{Verbatim}
\end{tcolorbox}
        
    Question \#1:

check the bottom 10 rows of data frame ``df''.

    \begin{tcolorbox}[breakable, size=fbox, boxrule=1pt, pad at break*=1mm,colback=cellbackground, colframe=cellborder]
\prompt{In}{incolor}{4}{\boxspacing}
\begin{Verbatim}[commandchars=\\\{\}]
\PY{c+c1}{\PYZsh{} Write your code below and press Shift+Enter to execute }
\PY{n}{df}\PY{o}{.}\PY{n}{tail}\PY{p}{(}\PY{l+m+mi}{10}\PY{p}{)}
\end{Verbatim}
\end{tcolorbox}

            \begin{tcolorbox}[breakable, size=fbox, boxrule=.5pt, pad at break*=1mm, opacityfill=0]
\prompt{Out}{outcolor}{4}{\boxspacing}
\begin{Verbatim}[commandchars=\\\{\}]
     0    1      2       3      4     5      6    7      8      9   {\ldots}   16  \textbackslash{}
195  -1   74  volvo     gas    std  four  wagon  rwd  front  104.3  {\ldots}  141
196  -2  103  volvo     gas    std  four  sedan  rwd  front  104.3  {\ldots}  141
197  -1   74  volvo     gas    std  four  wagon  rwd  front  104.3  {\ldots}  141
198  -2  103  volvo     gas  turbo  four  sedan  rwd  front  104.3  {\ldots}  130
199  -1   74  volvo     gas  turbo  four  wagon  rwd  front  104.3  {\ldots}  130
200  -1   95  volvo     gas    std  four  sedan  rwd  front  109.1  {\ldots}  141
201  -1   95  volvo     gas  turbo  four  sedan  rwd  front  109.1  {\ldots}  141
202  -1   95  volvo     gas    std  four  sedan  rwd  front  109.1  {\ldots}  173
203  -1   95  volvo  diesel  turbo  four  sedan  rwd  front  109.1  {\ldots}  145
204  -1   95  volvo     gas  turbo  four  sedan  rwd  front  109.1  {\ldots}  141

       17    18    19    20   21    22  23  24     25
195  mpfi  3.78  3.15   9.5  114  5400  23  28  13415
196  mpfi  3.78  3.15   9.5  114  5400  24  28  15985
197  mpfi  3.78  3.15   9.5  114  5400  24  28  16515
198  mpfi  3.62  3.15   7.5  162  5100  17  22  18420
199  mpfi  3.62  3.15   7.5  162  5100  17  22  18950
200  mpfi  3.78  3.15   9.5  114  5400  23  28  16845
201  mpfi  3.78  3.15   8.7  160  5300  19  25  19045
202  mpfi  3.58  2.87   8.8  134  5500  18  23  21485
203   idi  3.01  3.40  23.0  106  4800  26  27  22470
204  mpfi  3.78  3.15   9.5  114  5400  19  25  22625

[10 rows x 26 columns]
\end{Verbatim}
\end{tcolorbox}
        
    Question \#1 Answer:

Run the code below for the solution!

    Double-click here for the solution.

    Add Headers

Take a look at our dataset; pandas automatically set the header by an
integer from 0.

To better describe our data we can introduce a header, this information
is available at: https://archive.ics.uci.edu/ml/datasets/Automobile

Thus, we have to add headers manually.

Firstly, we create a list ``headers'' that include all column names in
order. Then, we use dataframe.columns = headers to replace the headers
by the list we created.

    \begin{tcolorbox}[breakable, size=fbox, boxrule=1pt, pad at break*=1mm,colback=cellbackground, colframe=cellborder]
\prompt{In}{incolor}{5}{\boxspacing}
\begin{Verbatim}[commandchars=\\\{\}]
\PY{c+c1}{\PYZsh{} create headers list}
\PY{n}{headers} \PY{o}{=} \PY{p}{[}\PY{l+s+s2}{\PYZdq{}}\PY{l+s+s2}{symboling}\PY{l+s+s2}{\PYZdq{}}\PY{p}{,}\PY{l+s+s2}{\PYZdq{}}\PY{l+s+s2}{normalized\PYZhy{}losses}\PY{l+s+s2}{\PYZdq{}}\PY{p}{,}\PY{l+s+s2}{\PYZdq{}}\PY{l+s+s2}{make}\PY{l+s+s2}{\PYZdq{}}\PY{p}{,}\PY{l+s+s2}{\PYZdq{}}\PY{l+s+s2}{fuel\PYZhy{}type}\PY{l+s+s2}{\PYZdq{}}\PY{p}{,}\PY{l+s+s2}{\PYZdq{}}\PY{l+s+s2}{aspiration}\PY{l+s+s2}{\PYZdq{}}\PY{p}{,} \PY{l+s+s2}{\PYZdq{}}\PY{l+s+s2}{num\PYZhy{}of\PYZhy{}doors}\PY{l+s+s2}{\PYZdq{}}\PY{p}{,}\PY{l+s+s2}{\PYZdq{}}\PY{l+s+s2}{body\PYZhy{}style}\PY{l+s+s2}{\PYZdq{}}\PY{p}{,}
         \PY{l+s+s2}{\PYZdq{}}\PY{l+s+s2}{drive\PYZhy{}wheels}\PY{l+s+s2}{\PYZdq{}}\PY{p}{,}\PY{l+s+s2}{\PYZdq{}}\PY{l+s+s2}{engine\PYZhy{}location}\PY{l+s+s2}{\PYZdq{}}\PY{p}{,}\PY{l+s+s2}{\PYZdq{}}\PY{l+s+s2}{wheel\PYZhy{}base}\PY{l+s+s2}{\PYZdq{}}\PY{p}{,} \PY{l+s+s2}{\PYZdq{}}\PY{l+s+s2}{length}\PY{l+s+s2}{\PYZdq{}}\PY{p}{,}\PY{l+s+s2}{\PYZdq{}}\PY{l+s+s2}{width}\PY{l+s+s2}{\PYZdq{}}\PY{p}{,}\PY{l+s+s2}{\PYZdq{}}\PY{l+s+s2}{height}\PY{l+s+s2}{\PYZdq{}}\PY{p}{,}\PY{l+s+s2}{\PYZdq{}}\PY{l+s+s2}{curb\PYZhy{}weight}\PY{l+s+s2}{\PYZdq{}}\PY{p}{,}\PY{l+s+s2}{\PYZdq{}}\PY{l+s+s2}{engine\PYZhy{}type}\PY{l+s+s2}{\PYZdq{}}\PY{p}{,}
         \PY{l+s+s2}{\PYZdq{}}\PY{l+s+s2}{num\PYZhy{}of\PYZhy{}cylinders}\PY{l+s+s2}{\PYZdq{}}\PY{p}{,} \PY{l+s+s2}{\PYZdq{}}\PY{l+s+s2}{engine\PYZhy{}size}\PY{l+s+s2}{\PYZdq{}}\PY{p}{,}\PY{l+s+s2}{\PYZdq{}}\PY{l+s+s2}{fuel\PYZhy{}system}\PY{l+s+s2}{\PYZdq{}}\PY{p}{,}\PY{l+s+s2}{\PYZdq{}}\PY{l+s+s2}{bore}\PY{l+s+s2}{\PYZdq{}}\PY{p}{,}\PY{l+s+s2}{\PYZdq{}}\PY{l+s+s2}{stroke}\PY{l+s+s2}{\PYZdq{}}\PY{p}{,}\PY{l+s+s2}{\PYZdq{}}\PY{l+s+s2}{compression\PYZhy{}ratio}\PY{l+s+s2}{\PYZdq{}}\PY{p}{,}\PY{l+s+s2}{\PYZdq{}}\PY{l+s+s2}{horsepower}\PY{l+s+s2}{\PYZdq{}}\PY{p}{,}
         \PY{l+s+s2}{\PYZdq{}}\PY{l+s+s2}{peak\PYZhy{}rpm}\PY{l+s+s2}{\PYZdq{}}\PY{p}{,}\PY{l+s+s2}{\PYZdq{}}\PY{l+s+s2}{city\PYZhy{}mpg}\PY{l+s+s2}{\PYZdq{}}\PY{p}{,}\PY{l+s+s2}{\PYZdq{}}\PY{l+s+s2}{highway\PYZhy{}mpg}\PY{l+s+s2}{\PYZdq{}}\PY{p}{,}\PY{l+s+s2}{\PYZdq{}}\PY{l+s+s2}{price}\PY{l+s+s2}{\PYZdq{}}\PY{p}{]}
\PY{n+nb}{print}\PY{p}{(}\PY{l+s+s2}{\PYZdq{}}\PY{l+s+s2}{headers}\PY{l+s+se}{\PYZbs{}n}\PY{l+s+s2}{\PYZdq{}}\PY{p}{,} \PY{n}{headers}\PY{p}{)}
\end{Verbatim}
\end{tcolorbox}

    \begin{Verbatim}[commandchars=\\\{\}]
headers
 ['symboling', 'normalized-losses', 'make', 'fuel-type', 'aspiration', 'num-of-
doors', 'body-style', 'drive-wheels', 'engine-location', 'wheel-base', 'length',
'width', 'height', 'curb-weight', 'engine-type', 'num-of-cylinders', 'engine-
size', 'fuel-system', 'bore', 'stroke', 'compression-ratio', 'horsepower',
'peak-rpm', 'city-mpg', 'highway-mpg', 'price']
    \end{Verbatim}

    We replace headers and recheck our data frame

    \begin{tcolorbox}[breakable, size=fbox, boxrule=1pt, pad at break*=1mm,colback=cellbackground, colframe=cellborder]
\prompt{In}{incolor}{6}{\boxspacing}
\begin{Verbatim}[commandchars=\\\{\}]
\PY{n}{df}\PY{o}{.}\PY{n}{columns} \PY{o}{=} \PY{n}{headers}
\PY{n}{df}\PY{o}{.}\PY{n}{head}\PY{p}{(}\PY{l+m+mi}{10}\PY{p}{)}
\end{Verbatim}
\end{tcolorbox}

            \begin{tcolorbox}[breakable, size=fbox, boxrule=.5pt, pad at break*=1mm, opacityfill=0]
\prompt{Out}{outcolor}{6}{\boxspacing}
\begin{Verbatim}[commandchars=\\\{\}]
   symboling normalized-losses         make fuel-type aspiration num-of-doors  \textbackslash{}
0          3                 ?  alfa-romero       gas        std          two
1          3                 ?  alfa-romero       gas        std          two
2          1                 ?  alfa-romero       gas        std          two
3          2               164         audi       gas        std         four
4          2               164         audi       gas        std         four
5          2                 ?         audi       gas        std          two
6          1               158         audi       gas        std         four
7          1                 ?         audi       gas        std         four
8          1               158         audi       gas      turbo         four
9          0                 ?         audi       gas      turbo          two

    body-style drive-wheels engine-location  wheel-base  {\ldots}  engine-size  \textbackslash{}
0  convertible          rwd           front        88.6  {\ldots}          130
1  convertible          rwd           front        88.6  {\ldots}          130
2    hatchback          rwd           front        94.5  {\ldots}          152
3        sedan          fwd           front        99.8  {\ldots}          109
4        sedan          4wd           front        99.4  {\ldots}          136
5        sedan          fwd           front        99.8  {\ldots}          136
6        sedan          fwd           front       105.8  {\ldots}          136
7        wagon          fwd           front       105.8  {\ldots}          136
8        sedan          fwd           front       105.8  {\ldots}          131
9    hatchback          4wd           front        99.5  {\ldots}          131

   fuel-system  bore  stroke compression-ratio horsepower  peak-rpm city-mpg  \textbackslash{}
0         mpfi  3.47    2.68               9.0        111      5000       21
1         mpfi  3.47    2.68               9.0        111      5000       21
2         mpfi  2.68    3.47               9.0        154      5000       19
3         mpfi  3.19    3.40              10.0        102      5500       24
4         mpfi  3.19    3.40               8.0        115      5500       18
5         mpfi  3.19    3.40               8.5        110      5500       19
6         mpfi  3.19    3.40               8.5        110      5500       19
7         mpfi  3.19    3.40               8.5        110      5500       19
8         mpfi  3.13    3.40               8.3        140      5500       17
9         mpfi  3.13    3.40               7.0        160      5500       16

  highway-mpg  price
0          27  13495
1          27  16500
2          26  16500
3          30  13950
4          22  17450
5          25  15250
6          25  17710
7          25  18920
8          20  23875
9          22      ?

[10 rows x 26 columns]
\end{Verbatim}
\end{tcolorbox}
        
    we can drop missing values along the column ``price'' as follows

    \begin{tcolorbox}[breakable, size=fbox, boxrule=1pt, pad at break*=1mm,colback=cellbackground, colframe=cellborder]
\prompt{In}{incolor}{7}{\boxspacing}
\begin{Verbatim}[commandchars=\\\{\}]
\PY{n}{df}\PY{o}{.}\PY{n}{dropna}\PY{p}{(}\PY{n}{subset}\PY{o}{=}\PY{p}{[}\PY{l+s+s2}{\PYZdq{}}\PY{l+s+s2}{price}\PY{l+s+s2}{\PYZdq{}}\PY{p}{]}\PY{p}{,} \PY{n}{axis}\PY{o}{=}\PY{l+m+mi}{0}\PY{p}{)}
\end{Verbatim}
\end{tcolorbox}

            \begin{tcolorbox}[breakable, size=fbox, boxrule=.5pt, pad at break*=1mm, opacityfill=0]
\prompt{Out}{outcolor}{7}{\boxspacing}
\begin{Verbatim}[commandchars=\\\{\}]
     symboling normalized-losses         make fuel-type aspiration  \textbackslash{}
0            3                 ?  alfa-romero       gas        std
1            3                 ?  alfa-romero       gas        std
2            1                 ?  alfa-romero       gas        std
3            2               164         audi       gas        std
4            2               164         audi       gas        std
..         {\ldots}               {\ldots}          {\ldots}       {\ldots}        {\ldots}
200         -1                95        volvo       gas        std
201         -1                95        volvo       gas      turbo
202         -1                95        volvo       gas        std
203         -1                95        volvo    diesel      turbo
204         -1                95        volvo       gas      turbo

    num-of-doors   body-style drive-wheels engine-location  wheel-base  {\ldots}  \textbackslash{}
0            two  convertible          rwd           front        88.6  {\ldots}
1            two  convertible          rwd           front        88.6  {\ldots}
2            two    hatchback          rwd           front        94.5  {\ldots}
3           four        sedan          fwd           front        99.8  {\ldots}
4           four        sedan          4wd           front        99.4  {\ldots}
..           {\ldots}          {\ldots}          {\ldots}             {\ldots}         {\ldots}  {\ldots}
200         four        sedan          rwd           front       109.1  {\ldots}
201         four        sedan          rwd           front       109.1  {\ldots}
202         four        sedan          rwd           front       109.1  {\ldots}
203         four        sedan          rwd           front       109.1  {\ldots}
204         four        sedan          rwd           front       109.1  {\ldots}

     engine-size  fuel-system  bore  stroke compression-ratio horsepower  \textbackslash{}
0            130         mpfi  3.47    2.68               9.0        111
1            130         mpfi  3.47    2.68               9.0        111
2            152         mpfi  2.68    3.47               9.0        154
3            109         mpfi  3.19    3.40              10.0        102
4            136         mpfi  3.19    3.40               8.0        115
..           {\ldots}          {\ldots}   {\ldots}     {\ldots}               {\ldots}        {\ldots}
200          141         mpfi  3.78    3.15               9.5        114
201          141         mpfi  3.78    3.15               8.7        160
202          173         mpfi  3.58    2.87               8.8        134
203          145          idi  3.01    3.40              23.0        106
204          141         mpfi  3.78    3.15               9.5        114

     peak-rpm city-mpg highway-mpg  price
0        5000       21          27  13495
1        5000       21          27  16500
2        5000       19          26  16500
3        5500       24          30  13950
4        5500       18          22  17450
..        {\ldots}      {\ldots}         {\ldots}    {\ldots}
200      5400       23          28  16845
201      5300       19          25  19045
202      5500       18          23  21485
203      4800       26          27  22470
204      5400       19          25  22625

[205 rows x 26 columns]
\end{Verbatim}
\end{tcolorbox}
        
    Now, we have successfully read the raw dataset and add the correct
headers into the data frame.

    Question \#2:

Find the name of the columns of the dataframe

    \begin{tcolorbox}[breakable, size=fbox, boxrule=1pt, pad at break*=1mm,colback=cellbackground, colframe=cellborder]
\prompt{In}{incolor}{11}{\boxspacing}
\begin{Verbatim}[commandchars=\\\{\}]
\PY{c+c1}{\PYZsh{} Write your code below and press Shift+Enter to execute }
\PY{n}{df}\PY{o}{.}\PY{n}{columns}
\end{Verbatim}
\end{tcolorbox}

    Double-click here for the solution.

    Save Dataset

Correspondingly, Pandas enables us to save the dataset to csv by using
the dataframe.to\_csv() method, you can add the file path and name along
with quotation marks in the brackets.

For example, if you would save the dataframe df as automobile.csv to
your local machine, you may use the syntax below:
df.to_csv("automobile.csv", index=False)
    We can also read and save other file formats, we can use similar
functions to \textbf{\texttt{pd.read\_csv()}} and
\textbf{\texttt{df.to\_csv()}} for other data formats, the functions are
listed in the following table:

    Read/Save Other Data Formats

\begin{longtable}[]{@{}lcr@{}}
\toprule
Data Formate & Read & Save\tabularnewline
\midrule
\endhead
csv & \texttt{pd.read\_csv()} & \texttt{df.to\_csv()}\tabularnewline
json & \texttt{pd.read\_json()} & \texttt{df.to\_json()}\tabularnewline
excel & \texttt{pd.read\_excel()} &
\texttt{df.to\_excel()}\tabularnewline
hdf & \texttt{pd.read\_hdf()} & \texttt{df.to\_hdf()}\tabularnewline
sql & \texttt{pd.read\_sql()} & \texttt{df.to\_sql()}\tabularnewline
\ldots{} & \ldots{} & \ldots{}\tabularnewline
\bottomrule
\end{longtable}

    Basic Insight of Dataset

After reading data into Pandas dataframe, it is time for us to explore
the dataset. There are several ways to obtain essential insights of the
data to help us better understand our dataset.

    Data Types

Data has a variety of types. The main types stored in Pandas dataframes
are object, float, int, bool and datetime64. In order to better learn
about each attribute, it is always good for us to know the data type of
each column. In Pandas:

    \begin{tcolorbox}[breakable, size=fbox, boxrule=1pt, pad at break*=1mm,colback=cellbackground, colframe=cellborder]
\prompt{In}{incolor}{12}{\boxspacing}
\begin{Verbatim}[commandchars=\\\{\}]
\PY{n}{df}\PY{o}{.}\PY{n}{dtypes}
\end{Verbatim}
\end{tcolorbox}

            \begin{tcolorbox}[breakable, size=fbox, boxrule=.5pt, pad at break*=1mm, opacityfill=0]
\prompt{Out}{outcolor}{12}{\boxspacing}
\begin{Verbatim}[commandchars=\\\{\}]
symboling              int64
normalized-losses     object
make                  object
fuel-type             object
aspiration            object
num-of-doors          object
body-style            object
drive-wheels          object
engine-location       object
wheel-base           float64
length               float64
width                float64
height               float64
curb-weight            int64
engine-type           object
num-of-cylinders      object
engine-size            int64
fuel-system           object
bore                  object
stroke                object
compression-ratio    float64
horsepower            object
peak-rpm              object
city-mpg               int64
highway-mpg            int64
price                 object
dtype: object
\end{Verbatim}
\end{tcolorbox}
        
    returns a Series with the data type of each column.

    \begin{tcolorbox}[breakable, size=fbox, boxrule=1pt, pad at break*=1mm,colback=cellbackground, colframe=cellborder]
\prompt{In}{incolor}{13}{\boxspacing}
\begin{Verbatim}[commandchars=\\\{\}]
\PY{c+c1}{\PYZsh{} check the data type of data frame \PYZdq{}df\PYZdq{} by .dtypes}
\PY{n+nb}{print}\PY{p}{(}\PY{n}{df}\PY{o}{.}\PY{n}{dtypes}\PY{p}{)}
\end{Verbatim}
\end{tcolorbox}

    \begin{Verbatim}[commandchars=\\\{\}]
symboling              int64
normalized-losses     object
make                  object
fuel-type             object
aspiration            object
num-of-doors          object
body-style            object
drive-wheels          object
engine-location       object
wheel-base           float64
length               float64
width                float64
height               float64
curb-weight            int64
engine-type           object
num-of-cylinders      object
engine-size            int64
fuel-system           object
bore                  object
stroke                object
compression-ratio    float64
horsepower            object
peak-rpm              object
city-mpg               int64
highway-mpg            int64
price                 object
dtype: object
    \end{Verbatim}

    As a result, as shown above, it is clear to see that the data type of
``symboling'' and ``curb-weight'' are int64, ``normalized-losses'' is
object, and ``wheel-base'' is float64, etc.

These data types can be changed; we will learn how to accomplish this in
a later module.

    Describe

If we would like to get a statistical summary of each column, such as
count, column mean value, column standard deviation, etc. We use the
describe method:
dataframe.describe()
    This method will provide various summary statistics, excluding NaN (Not
a Number) values.

    \begin{tcolorbox}[breakable, size=fbox, boxrule=1pt, pad at break*=1mm,colback=cellbackground, colframe=cellborder]
\prompt{In}{incolor}{14}{\boxspacing}
\begin{Verbatim}[commandchars=\\\{\}]
\PY{n}{df}\PY{o}{.}\PY{n}{describe}\PY{p}{(}\PY{p}{)}
\end{Verbatim}
\end{tcolorbox}

            \begin{tcolorbox}[breakable, size=fbox, boxrule=.5pt, pad at break*=1mm, opacityfill=0]
\prompt{Out}{outcolor}{14}{\boxspacing}
\begin{Verbatim}[commandchars=\\\{\}]
        symboling  wheel-base      length       width      height  \textbackslash{}
count  205.000000  205.000000  205.000000  205.000000  205.000000
mean     0.834146   98.756585  174.049268   65.907805   53.724878
std      1.245307    6.021776   12.337289    2.145204    2.443522
min     -2.000000   86.600000  141.100000   60.300000   47.800000
25\%      0.000000   94.500000  166.300000   64.100000   52.000000
50\%      1.000000   97.000000  173.200000   65.500000   54.100000
75\%      2.000000  102.400000  183.100000   66.900000   55.500000
max      3.000000  120.900000  208.100000   72.300000   59.800000

       curb-weight  engine-size  compression-ratio    city-mpg  highway-mpg
count   205.000000   205.000000         205.000000  205.000000   205.000000
mean   2555.565854   126.907317          10.142537   25.219512    30.751220
std     520.680204    41.642693           3.972040    6.542142     6.886443
min    1488.000000    61.000000           7.000000   13.000000    16.000000
25\%    2145.000000    97.000000           8.600000   19.000000    25.000000
50\%    2414.000000   120.000000           9.000000   24.000000    30.000000
75\%    2935.000000   141.000000           9.400000   30.000000    34.000000
max    4066.000000   326.000000          23.000000   49.000000    54.000000
\end{Verbatim}
\end{tcolorbox}
        
    This shows the statistical summary of all numeric-typed (int, float)
columns. For example, the attribute ``symboling'' has 205 counts, the
mean value of this column is 0.83, the standard deviation is 1.25, the
minimum value is -2, 25th percentile is 0, 50th percentile is 1, 75th
percentile is 2, and the maximum value is 3. However, what if we would
also like to check all the columns including those that are of type
object.

You can add an argument include = ``all'' inside the bracket. Let's try
it again.

    \begin{tcolorbox}[breakable, size=fbox, boxrule=1pt, pad at break*=1mm,colback=cellbackground, colframe=cellborder]
\prompt{In}{incolor}{15}{\boxspacing}
\begin{Verbatim}[commandchars=\\\{\}]
\PY{c+c1}{\PYZsh{} describe all the columns in \PYZdq{}df\PYZdq{} }
\PY{n}{df}\PY{o}{.}\PY{n}{describe}\PY{p}{(}\PY{n}{include} \PY{o}{=} \PY{l+s+s2}{\PYZdq{}}\PY{l+s+s2}{all}\PY{l+s+s2}{\PYZdq{}}\PY{p}{)}
\end{Verbatim}
\end{tcolorbox}

            \begin{tcolorbox}[breakable, size=fbox, boxrule=.5pt, pad at break*=1mm, opacityfill=0]
\prompt{Out}{outcolor}{15}{\boxspacing}
\begin{Verbatim}[commandchars=\\\{\}]
         symboling normalized-losses    make fuel-type aspiration  \textbackslash{}
count   205.000000               205     205       205        205
unique         NaN                52      22         2          2
top            NaN                 ?  toyota       gas        std
freq           NaN                41      32       185        168
mean      0.834146               NaN     NaN       NaN        NaN
std       1.245307               NaN     NaN       NaN        NaN
min      -2.000000               NaN     NaN       NaN        NaN
25\%       0.000000               NaN     NaN       NaN        NaN
50\%       1.000000               NaN     NaN       NaN        NaN
75\%       2.000000               NaN     NaN       NaN        NaN
max       3.000000               NaN     NaN       NaN        NaN

       num-of-doors body-style drive-wheels engine-location  wheel-base  {\ldots}  \textbackslash{}
count           205        205          205             205  205.000000  {\ldots}
unique            3          5            3               2         NaN  {\ldots}
top            four      sedan          fwd           front         NaN  {\ldots}
freq            114         96          120             202         NaN  {\ldots}
mean            NaN        NaN          NaN             NaN   98.756585  {\ldots}
std             NaN        NaN          NaN             NaN    6.021776  {\ldots}
min             NaN        NaN          NaN             NaN   86.600000  {\ldots}
25\%             NaN        NaN          NaN             NaN   94.500000  {\ldots}
50\%             NaN        NaN          NaN             NaN   97.000000  {\ldots}
75\%             NaN        NaN          NaN             NaN  102.400000  {\ldots}
max             NaN        NaN          NaN             NaN  120.900000  {\ldots}

        engine-size  fuel-system  bore  stroke compression-ratio horsepower  \textbackslash{}
count    205.000000          205   205     205        205.000000        205
unique          NaN            8    39      37               NaN         60
top             NaN         mpfi  3.62    3.40               NaN         68
freq            NaN           94    23      20               NaN         19
mean     126.907317          NaN   NaN     NaN         10.142537        NaN
std       41.642693          NaN   NaN     NaN          3.972040        NaN
min       61.000000          NaN   NaN     NaN          7.000000        NaN
25\%       97.000000          NaN   NaN     NaN          8.600000        NaN
50\%      120.000000          NaN   NaN     NaN          9.000000        NaN
75\%      141.000000          NaN   NaN     NaN          9.400000        NaN
max      326.000000          NaN   NaN     NaN         23.000000        NaN

        peak-rpm    city-mpg highway-mpg price
count        205  205.000000  205.000000   205
unique        24         NaN         NaN   187
top         5500         NaN         NaN     ?
freq          37         NaN         NaN     4
mean         NaN   25.219512   30.751220   NaN
std          NaN    6.542142    6.886443   NaN
min          NaN   13.000000   16.000000   NaN
25\%          NaN   19.000000   25.000000   NaN
50\%          NaN   24.000000   30.000000   NaN
75\%          NaN   30.000000   34.000000   NaN
max          NaN   49.000000   54.000000   NaN

[11 rows x 26 columns]
\end{Verbatim}
\end{tcolorbox}
        
    Now, it provides the statistical summary of all the columns, including
object-typed attributes. We can now see how many unique values, which is
the top value and the frequency of top value in the object-typed
columns. Some values in the table above show as ``NaN'', this is because
those numbers are not available regarding a particular column type.

    Question \#3:

You can select the columns of a data frame by indicating the name of
each column, for example, you can select the three columns as follows:

dataframe{[}{[}' column 1 `,column 2', `column 3'{]}{]}

Where ``column'' is the name of the column, you can apply the method
``.describe()'' to get the statistics of those columns as follows:

dataframe{[}{[}' column 1 `,column 2', `column 3'{]} {]}.describe()

Apply the method to ``.describe()'' to the columns `length' and
`compression-ratio'.

    \begin{tcolorbox}[breakable, size=fbox, boxrule=1pt, pad at break*=1mm,colback=cellbackground, colframe=cellborder]
\prompt{In}{incolor}{22}{\boxspacing}
\begin{Verbatim}[commandchars=\\\{\}]
\PY{c+c1}{\PYZsh{} Write your code below and press Shift+Enter to execute }
\PY{n}{df}\PY{p}{[}\PY{p}{[}\PY{l+s+s1}{\PYZsq{}}\PY{l+s+s1}{compression\PYZhy{}ratio}\PY{l+s+s1}{\PYZsq{}}\PY{p}{,} \PY{l+s+s1}{\PYZsq{}}\PY{l+s+s1}{length}\PY{l+s+s1}{\PYZsq{}}\PY{p}{]} \PY{p}{]}\PY{o}{.}\PY{n}{describe}\PY{p}{(}\PY{p}{)}
\end{Verbatim}
\end{tcolorbox}

            \begin{tcolorbox}[breakable, size=fbox, boxrule=.5pt, pad at break*=1mm, opacityfill=0]
\prompt{Out}{outcolor}{22}{\boxspacing}
\begin{Verbatim}[commandchars=\\\{\}]
       compression-ratio      length
count         205.000000  205.000000
mean           10.142537  174.049268
std             3.972040   12.337289
min             7.000000  141.100000
25\%             8.600000  166.300000
50\%             9.000000  173.200000
75\%             9.400000  183.100000
max            23.000000  208.100000
\end{Verbatim}
\end{tcolorbox}
        
    Double-click here for the solution.

    Info

Another method you can use to check your dataset is:
dataframe.info
    It provide a concise summary of your DataFrame.

    \begin{tcolorbox}[breakable, size=fbox, boxrule=1pt, pad at break*=1mm,colback=cellbackground, colframe=cellborder]
\prompt{In}{incolor}{23}{\boxspacing}
\begin{Verbatim}[commandchars=\\\{\}]
\PY{c+c1}{\PYZsh{} look at the info of \PYZdq{}df\PYZdq{}}
\PY{n}{df}\PY{o}{.}\PY{n}{info}
\end{Verbatim}
\end{tcolorbox}

            \begin{tcolorbox}[breakable, size=fbox, boxrule=.5pt, pad at break*=1mm, opacityfill=0]
\prompt{Out}{outcolor}{23}{\boxspacing}
\begin{Verbatim}[commandchars=\\\{\}]
<bound method DataFrame.info of      symboling normalized-losses         make
fuel-type aspiration  \textbackslash{}
0            3                 ?  alfa-romero       gas        std
1            3                 ?  alfa-romero       gas        std
2            1                 ?  alfa-romero       gas        std
3            2               164         audi       gas        std
4            2               164         audi       gas        std
..         {\ldots}               {\ldots}          {\ldots}       {\ldots}        {\ldots}
200         -1                95        volvo       gas        std
201         -1                95        volvo       gas      turbo
202         -1                95        volvo       gas        std
203         -1                95        volvo    diesel      turbo
204         -1                95        volvo       gas      turbo

    num-of-doors   body-style drive-wheels engine-location  wheel-base  {\ldots}  \textbackslash{}
0            two  convertible          rwd           front        88.6  {\ldots}
1            two  convertible          rwd           front        88.6  {\ldots}
2            two    hatchback          rwd           front        94.5  {\ldots}
3           four        sedan          fwd           front        99.8  {\ldots}
4           four        sedan          4wd           front        99.4  {\ldots}
..           {\ldots}          {\ldots}          {\ldots}             {\ldots}         {\ldots}  {\ldots}
200         four        sedan          rwd           front       109.1  {\ldots}
201         four        sedan          rwd           front       109.1  {\ldots}
202         four        sedan          rwd           front       109.1  {\ldots}
203         four        sedan          rwd           front       109.1  {\ldots}
204         four        sedan          rwd           front       109.1  {\ldots}

     engine-size  fuel-system  bore  stroke compression-ratio horsepower  \textbackslash{}
0            130         mpfi  3.47    2.68               9.0        111
1            130         mpfi  3.47    2.68               9.0        111
2            152         mpfi  2.68    3.47               9.0        154
3            109         mpfi  3.19    3.40              10.0        102
4            136         mpfi  3.19    3.40               8.0        115
..           {\ldots}          {\ldots}   {\ldots}     {\ldots}               {\ldots}        {\ldots}
200          141         mpfi  3.78    3.15               9.5        114
201          141         mpfi  3.78    3.15               8.7        160
202          173         mpfi  3.58    2.87               8.8        134
203          145          idi  3.01    3.40              23.0        106
204          141         mpfi  3.78    3.15               9.5        114

     peak-rpm city-mpg highway-mpg  price
0        5000       21          27  13495
1        5000       21          27  16500
2        5000       19          26  16500
3        5500       24          30  13950
4        5500       18          22  17450
..        {\ldots}      {\ldots}         {\ldots}    {\ldots}
200      5400       23          28  16845
201      5300       19          25  19045
202      5500       18          23  21485
203      4800       26          27  22470
204      5400       19          25  22625

[205 rows x 26 columns]>
\end{Verbatim}
\end{tcolorbox}
        
    Here we are able to see the information of our dataframe, with the top
30 rows and the bottom 30 rows. And, it also shows us the whole data
frame has 205 rows and 26 columns in total.

    Excellent! You have just completed the Introduction Notebook!

    \begin{verbatim}
<p><a href="https://cocl.us/corsera_da0101en_notebook_bottom"><img src="https://s3-api.us-geo.objectstorage.softlayer.net/cf-courses-data/CognitiveClass/DA0101EN/Images/BottomAd.png" width="750" align="center"></a></p>
\end{verbatim}

    About the Authors:

This notebook was written by Mahdi Noorian PhD, Joseph Santarcangelo,
Bahare Talayian, Eric Xiao, Steven Dong, Parizad, Hima Vsudevan and
Fiorella Wenver and Yi Yao.

Joseph Santarcangelo is a Data Scientist at IBM, and holds a PhD in
Electrical Engineering. His research focused on using Machine Learning,
Signal Processing, and Computer Vision to determine how videos impact
human cognition. Joseph has been working for IBM since he completed his
PhD.

    Copyright © 2018 IBM Developer Skills Network. This notebook and its
source code are released under the terms of the MIT License.


    % Add a bibliography block to the postdoc
    
    
    
\end{document}
